\documentclass[12pt,preprintnumbers,amsmath,amssymb,titlepage]{report}
\usepackage{amsmath,amsthm,amssymb,dcolumn,epsf,ulem,relsize,graphicx}
\setkeys{Gin}{width=8.6cm,keepaspectratio}
\graphicspath{{pngs/}}
\usepackage{dcolumn}% Align table columns on decimal point
\usepackage{bm}% bold math
\usepackage[]{units}
\usepackage{subfig}
\usepackage{floatrow}
\floatsetup[table]{capposition=bottom} 
\captionsetup{font={small,rm}}
\captionsetup{belowskip=0pt}
\usepackage{wrapfig}
\usepackage{caption}
\usepackage{array}
\usepackage{booktabs}
\usepackage{verbatim}
\usepackage[margin = 1in]{geometry}
\usepackage{siunitx}
\sisetup{output-exponent-marker=\textsc{e}}
\usepackage{setspace}
\onehalfspace
\setlength{\parindent}{0em}
\setlength{\parskip}{.5em}
\allowdisplaybreaks[2]
\newcommand{\refe}[1]{Eq. (\ref{#1})}


\begin{document}

\title{Modeling the Lorenz Attractor with SINDY}
\author{Lia Gianfortone}
\date{December 2017}

\maketitle


\section*{Abstract}

\section*{Introduction}

System identification techniques that use measured data to recover a system's governing equations are useful for a wide range of applications. The system-identification algorithm proposed in [paper] involves sparse identifications of nonlinear dynamical systems (SINDy). The algorithm relies on the assumption that dynamics of the system depend on only a few linear and nonlinear terms which will guarantee the sparsity of the solution. 


\section*{SINDY Algorithm}
The algorithm is effective under the assumption that the systems studied with SINDy have dynamics that are depend by only a few terms and are thus sparse in a high-dimensional, nonlinear function space. 

We consider systems of the form 
\begin{equation}
\frac{d}{dt}\bm{x}(t) = \dot{x} = \bm{f}(\bm{x}(t))
\end{equation}
where $\bm{x}(t) \in \mathbb{R}^n$ is the state of the system at time $t$ and $\bm{f}$ consists of the governing equations of the system. 

Given $m$ (potentially noisy) measurements of the state $\bm{x}$ and its derivative $\bm{\dot{x}}$ over time, we arrange them into data matrices $\bm{X}$ and $\bm{\dot{X}}$ as follows
\begin{equation*}
	\bm{X} = \begin{bmatrix} 
				\bm{x}^T(t_1) \\ \bm{x}^T(t_2) \\ \vdots \\ \bm{x}^T(t_m)
			\end{bmatrix}
		   = \begin{bmatrix}
		   		x_1(t_1) & x_2(t_1) & \cdots & x_n(t_1) \\
		   		x_1(t_2) & x_2(t_2) & \cdots & x_n(t_2) \\
		   		\vdots   & \vdots   & \ddots & \vdots   \\
		   		x_1(t_m) & x_2(t_m) & \cdots & x_n(t_m) 
	   		\end{bmatrix}
\end{equation*}

\begin{equation*}
	\bm{\dot{X}} = \begin{bmatrix} 
				\bm{\dot{x}}^T(t_1) \\ \bm{\dot{x}}^T(t_2) \\ \vdots \\ \bm{\dot{x}}^T(t_m)
			\end{bmatrix}
		   = \begin{bmatrix}
		   		\dot{x}_1(t_1) & \dot{x}_2(t_1) & \cdots & \dot{x}_n(t_1) \\
		   		\dot{x}_1(t_2) & \dot{x}_2(t_2) & \cdots & \dot{x}_n(t_2) \\
		   		\vdots    	   & \vdots   		& \ddots & \vdots  \\
		   		\dot{x}_1(t_m) & \dot{x}_2(t_m) & \cdots & \dot{x}_n(t_m)
	   		\end{bmatrix}.
\end{equation*}

To identify the active terms in the dynamics, we construct the system 
\begin{equation} \label{eqn:main}
	\bm{\dot{X}} = \bm{\Theta}(\bm{X})\bm{\Xi}
\end{equation}
where the state data are input to the library of linear and nonlinear candidate functions, $\bm{\Theta}(\bm{X})$, which consists of constant, polynomial, and trigonometric terms that may be chosen based on hypotheses (based on symmetry, physics, etc.) about the system dynamics. [[More about choosing functions]]. The library of functions has the form
\begin{equation}
	\bm{\Theta}(\bm{X}) = 
	\begin{bmatrix}
		\mid & \mid 	& \mid 	 		&        & \mid	  		& \mid 		   &  		\\
		1	 & \bm{X}   & \bm{X}^{P_2}  & \cdots & \sin(\bm{X}) & \cos(\bm{X}) & \cdots \\
		\mid & \mid 	& \mid 	 		&        & \mid	  		& \mid  	   &
	\end{bmatrix}
\end{equation}
where polynomial cross-terms of degree $i$ are denoted $\bm{X}^{P_i}$. For example, $\bm{X}^{P_i}$ contains quadratic nonlinearities in the state $\bm{x}$,
\begin{equation*}
	\bm{X}^{P_2} = 
		\begin{bmatrix}
			x_1^2(t_1)	&	x_1(t_1)x_2(t_1) & \cdots & x_2^2(t_1) & \cdots & x_n^2(t_1) \\
			x_1^2(t_2)	&	x_1(t_2)x_2(t_2) & \cdots & x_2^2(t_2) & \cdots & x_n^2(t_2) \\
			\vdots		&	\vdots		     & \ddots & \vdots	   & \ddots & \vdots	 \\
			x_1^2(t_m)	&	x_1(t_m)x_2(t_m) & \cdots & x_2^2(t_m) & \cdots & x_n^2(t_m) \\
		\end{bmatrix}.
\end{equation*}
The final term in \refe{eqn:main}, $\bm{\Xi} = [\bm{\xi}_1 \bm{\xi}_2 \cdots \bm{\xi}_n]$ is the desired sparse matrix of coefficients that identifies which of the candidate functions in $\bm{\Theta}(\bm{X})$ are active in the system. 

Solving for $\bm{\Xi}$ requires a distinct optimization of each column, corresponding to each of the $n$ dynamical equations that guide the system. 
Solving for these coefficients requires a distinct optimization for each vector equation, 
\begin{equation} \label{eqn:vector}
	\bm{\dot{x}} = \bm{f}(\bm{x}) = \bm{\Xi}^T(\bm{\Theta}(\bm{x}^T))^T.
\end{equation}

Algorithms to perform the desired sparse regression of \refe{eqn:vector} include LASSO, which uses the $L_1$-norm to ensure sparsity, and the sequential least squares method. In this study, sequential least squares was the method of choice.





\section*{Lorenz Attractor}
The Lorenz system is an interesting test for the SINDy method. The known dynamics are 
\begin{align}
	\dot{x} &= \sigma(y - x) \\
	\dot{y} &= x(\rho - z) - y \\
	\dot{z} &= xy - \beta * z.
\end{align}
This system demonstrates chaotic behavior under certain conditions and is observably sparse in a Cartesian function space with only two nonlinear terms, $xz$ in $\dot{y}$ and $xy$ in $\dot{z}$. 

Data is generated in Matlab using the \verb|ode45| function and the chaotic parameters
\begin{equation}
	\sigma=10 \qquad \beta=8/3 \qquad \rho=28.
\end{equation} 
Random Gaussian noise is added to the $\bm{\dot{X}}$ data to challenge the algorithm. Figure \ref{fig:Lorenz} shows the solution generated by \verb|ode45| beside SINDy's interpretation of the system. [[Errors]]


\section*{SINDy for Discrete Systems}
The SINDy method can be adapted to determine the dynamics governing discrete systems. Systems 
\begin{equation} \label{discrete}
	\bm{x}_{k+1} = \bm{f}(\bm{x}_k)
\end{equation}
are particularly well suited for the SINDy algorithm due to the absence of errors from the measurement or generation of state derivate data. The $m$ data points can be arranged in the two matrices 
\begin{equation}
	\bm{X}_1^{m-1} = 
		\begin{bmatrix}
			\bm{x_1^T} \\ \bm{x_2^T} \\ \vdots \\ \bm{x_{m-1}^T}
		\end{bmatrix}
	\bm{X}_2^{m} = 
		\begin{bmatrix}
			\bm{x_2^T} \\ \bm{x_3^T} \\ \vdots \\ \bm{x_{m}^T}
		\end{bmatrix}.
\end{equation}
Identifying $\bm{f}(\bm{x}_k)$ in \refe{discrete} with SINDy consists of selecting a function basis $\bm{\Theta}(\bm{x}^T)$ and constructing the relation
\begin{equation}
	\bm{X_2^m} = \bm{\Theta}(\bm{X_1^{m-1}})\Xi
\end{equation}
so that 
\begin{equation}
	\bm{f}(\bm{x}_k) = \Xi^T \bm{\Theta}(\bm{x}^T)^T
\end{equation}
and solving for $\Xi$ with column-by-column optimization. 



\section*{Duffing Map}


One discrete, nonlinear chaotic relation of interest is the Duffing map,
\begin{align}
	x_{k+1} &= y_k \\ 
	y_{k+1} &= -\beta x_k + \alpha y_k - y_k^3.
\end{align}
Choosing $\alpha = 2.75$ and $\beta=0.2$ guarantees chaotic behavior in the system and our satisfaction in the problem as a test of SINDy for discrete systems. An exact set of solutions is then given by
\begin{equation}
	\bm{f}(\bm{x}) = \Xi^T
\end{equation}

\section*{SINDy for Systems with Forcing}

\section*{External Forcing}
Identifying the dynamics of systems with purely external forcing is a straightforward procedure if the control measurements are known. The control vectors are entered in a data matrix $\bm{Y}$ as
\begin{equation}
	\bm{Y} = 
		\begin{bmatrix}
			u(t_1)^T \\ u(t_2)^T \\ \vdots \\ u(t_m)^T
		\end{bmatrix}
\end{equation}
and the selection of basis functions in $\bm{Theta}$ is expanded to include control terms including cross-terms with the state variables. Optimization is then performed on the columns of 
\begin{equation}
		\bm{\dot{X}} = \bm{\Theta}(\bm{X},\bm{Y})\bm{\Xi}
\end{equation}
to determine a sparse $\Xi$ that identifies the dynamics of the system regardless regardless of the control that is acting on it, so long as the control is itself not a function of the state.

\section*{Duffing Equation}
The Duffing equation describes a second order nonlinear differential equation with external and sinusoidal forcing. The equation models damped and driven oscillators including the motion of a classical particle in a double well potential. The equation is
\begin{equation}
	\ddot x + \delta \dot x + \alpha x + \beta x^3 = \gamma \cos(\omega t).
\end{equation}
Systems governed by the Duffing equation will exhibit chaotic behavior for certain values of $\alpha,\ \beta,\ \gamma,$ and $\delta$.
Up until this point, we have not considered using SINDy to determine coefficients in higher order differential equations that govern a measurable system. This is no obstacle as we recall that any (??) $p$-th order differential equation can be separated in to $p$ first order differential equations. As such, the Duffing equation can be rewritten as 
\begin{align}
	\dot x &= v \\
	\dot v &= x - x^3 - \gamma v + \delta \cos(\omega t).
\end{align}
The dynamics can then be recovered with 
\begin{equation}
		\bm{[\dot{X}\ \dot{V}]} = \bm{\Theta}(\bm{[X\ V]},\bm{Y})\bm{\Xi}.
\end{equation}



\section*{Lorenz Equations with Forcing}

\section*{SINDyC Algorithm}

\section*{Possible Problems}
- wrong candidate functions
- coefficients of dynamics are smaller than S.L.S. threshold, $\lambda$
- noisiness of data


Given that computation of $\bm{\Theta}(\bm{X})$ grows factorially with $n$, the dimension of the state, this  method is less preferable than dynamic mode decomposition (DMD) which uses single value and eigen- decomposition to identify the normal modes of linear systems. Thus if a system is expected or known to be linear, the DMD method would likely be a better choice of system identification.





\section{Further Study}

\section*{Conclusion}

\section*{Bibliography}

\appendix{Code}

\appendix{Figures}



\end{document}